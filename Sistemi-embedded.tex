\documentclass[12pt]{article}
\usepackage[utf8]{inputenc}
\usepackage{graphicx} % Per includere immagini
\usepackage{titling}  % Per personalizzare lo spazio del titolo
\usepackage{ccicons}  % Per i simboli Creative Commons
\usepackage{geometry} % Per personalizzare i margini
\usepackage{hyperref} % Per i link
\usepackage{pgffor}
\usepackage{listings}
\usepackage{color}

\definecolor{codegreen}{rgb}{0,0.6,0}
\definecolor{codegray}{rgb}{0.5,0.5,0.5}
\definecolor{codepurple}{rgb}{0.58,0,0.82}
\definecolor{backcolour}{rgb}{0.95,0.95,0.92}

\lstdefinestyle{mystyle}{
    backgroundcolor=\color{backcolour},   
    commentstyle=\color{codegreen},
    keywordstyle=\color{magenta},
    numberstyle=\tiny\color{codegray},
    stringstyle=\color{codepurple},
    basicstyle=\ttfamily\footnotesize,
    breakatwhitespace=false,         
    breaklines=true,                 
    captionpos=b,                    
    keepspaces=true,                 
    numbers=left,                    
    numbersep=5pt,                  
    showspaces=false,                
    showstringspaces=false,
    showtabs=false,                  
    tabsize=2
}

\lstset{style=mystyle}

% Imposta i margini della pagina
\geometry{
  top=2cm,
  bottom=2cm,
  left=3cm,
  right=3cm
}

\lstset{
  basicstyle=\ttfamily,
  breaklines=true,
  columns=fullflexible
}

\setlength{\droptitle}{-5em} % Sposta in alto il titolo

\title{
    \Large \textbf{UNIVERSITA' DI SALERNO} \\[0.5em]
    \small DIPARTIMENTO DI INGEGNERIA DELL'INFORMAZIONE ED ELETTRICA E MATEMATICA APPLICATA\\[5em]
    \includegraphics[width=0.6\textwidth]{logo_uni.png}\\[3em] % Logo dell'Università
    \normalsize Laurea Magistrale in Ingegneria Informatica \\[1em]
    \Large \textbf {Project work} \\[1em]
    \large \textbf {Deliverable 1} \\ [1em]
    \large {Sistemi Embedded} \\[1em]
}

\author{
    \textbf{Gruppo: 8} \\
    \normalsize Marotta Giuseppe - 0622702302 - g.marotta@studenti.unisa.it\\
    \normalsize Rea Gaetano - 0622702190 - g.rea7@studenti.unisa.it\\
    \normalsize Squitieri Giuseppe - 0622702339 - g.squitieri8@studenti.unisa.it\\ 
    \normalsize Tramice Davide - 0622702194 - d.tramice@studenti.unisa.it\\ \\
    }

\date{
    ANNO ACCADEMICO 2023/2024 % Data
}

\begin{document}

\begin{titlingpage} % Crea una pagina di titolo personalizzata
\maketitle
\thispagestyle{empty} % Rimuove il numero di pagina
%\begin{center}
    %\ccbyncnd % Simbolo Creative Commons
%\end{center}
\end{titlingpage}

% Creazione di una nuova pagina
\newpage

% Aggiungi l'indice delle sezioni
\tableofcontents

\newpage

\section{Progettazione del sistema}
Testo della sezione 1.
\subsection{User stories}
testo user stories
\subsubsection{US1:}
Come utente, voglio essere in grado di aprire il cancello premendo il pulsante B1.
\begin{itemize}
    \item Criteri di accettazione
    \begin{enumerate}
        \item Il cancello si apre quando l'utente preme il pulsante B1.
        \item Se il cancello è in fase di apertura o è già aperto, premere il pulsante B1 attiva la fase di chiusura.
    \end{enumerate}
\end{itemize}
\subsubsection{US2:}
Come utente, voglio poter chiudere il cancello premendo il pulsante B1.
\begin{itemize}
    \item Criteri di accettazione
    \begin{enumerate}
        \item Il cancello si chiude quando l'utente preme il pulsante B1 mentre è aperto o in fase di apertura.
        \item Se il cancello è in fase di chiusura o è già chiuso, premere il pulsante B1 attiva la fase di apertura.
    \end{enumerate}
\end{itemize}
\begin{itemize}
    \item Criteri di accettazione
\end{itemize}
\subsubsection{US3:}
testo user stories 3
\begin{itemize}
    \item Criteri di accettazione
\end{itemize}
\subsubsection{US4:}
testo user stories 4
\begin{itemize}
    \item Criteri di accettazione
\end{itemize}
\subsubsection{US5:}
testo user stories 5
\begin{itemize}
    \item Criteri di accettazione
\end{itemize}
\subsubsection{US6:}
testo user stories 6
\begin{itemize}
    \item Criteri di accettazione
\end{itemize}
\subsubsection{US7:}
testo user stories 7
\begin{itemize}
    \item Criteri di accettazione
\end{itemize}
\subsubsection{US8:}
testo user stories 8
\begin{itemize}
    \item Criteri di accettazione
\end{itemize}
\subsubsection{US9:}
testo user stories 9
\begin{itemize}
    \item Criteri di accettazione
\end{itemize}
\subsubsection{US10:}
testo user stories 10



\subsection{Use case diagram}
Testo del' Use case diagram.

\subsection{Activity diagrams}
Testo dell'Activity diagram.
\subsubsection{Scenario 1}
Testo dello scenario 1
\subsubsection{Scenario 2}
Testo dello scenario 2
\subsubsection{Scenario 3}
Testo dello scenario 3
\subsubsection{Scenario 4}
Testo dello scenario 4
\subsubsection{Scenario 5}
Testo dello scenario 5
\subsubsection{Scenario 6}
Testo dello scenario 6
\subsubsection{Scenario 7}
Testo dello scenario 7

\subsection{State diagram}
Testo dello state diagram.

\section*{Elenco delle figure}
\addcontentsline{toc}{section}{Elenco delle figure}
\end{document}
